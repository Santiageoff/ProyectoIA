% Options for packages loaded elsewhere
\PassOptionsToPackage{unicode}{hyperref}
\PassOptionsToPackage{hyphens}{url}
%
\documentclass[
]{article}
\usepackage{amsmath,amssymb}
\usepackage{lmodern}
\usepackage{iftex}
\ifPDFTeX
  \usepackage[T1]{fontenc}
  \usepackage[utf8]{inputenc}
  \usepackage{textcomp} % provide euro and other symbols
\else % if luatex or xetex
  \usepackage{unicode-math}
  \defaultfontfeatures{Scale=MatchLowercase}
  \defaultfontfeatures[\rmfamily]{Ligatures=TeX,Scale=1}
\fi
% Use upquote if available, for straight quotes in verbatim environments
\IfFileExists{upquote.sty}{\usepackage{upquote}}{}
\IfFileExists{microtype.sty}{% use microtype if available
  \usepackage[]{microtype}
  \UseMicrotypeSet[protrusion]{basicmath} % disable protrusion for tt fonts
}{}
\makeatletter
\@ifundefined{KOMAClassName}{% if non-KOMA class
  \IfFileExists{parskip.sty}{%
    \usepackage{parskip}
  }{% else
    \setlength{\parindent}{0pt}
    \setlength{\parskip}{6pt plus 2pt minus 1pt}}
}{% if KOMA class
  \KOMAoptions{parskip=half}}
\makeatother
\usepackage{xcolor}
\IfFileExists{xurl.sty}{\usepackage{xurl}}{} % add URL line breaks if available
\IfFileExists{bookmark.sty}{\usepackage{bookmark}}{\usepackage{hyperref}}
\hypersetup{
  pdftitle={Predicción de Rendimiento Estudiantil},
  hidelinks,
  pdfcreator={LaTeX via pandoc}}
\urlstyle{same} % disable monospaced font for URLs
\usepackage{longtable,booktabs,array}
\usepackage{calc} % for calculating minipage widths
% Correct order of tables after \paragraph or \subparagraph
\usepackage{etoolbox}
\makeatletter
\patchcmd\longtable{\par}{\if@noskipsec\mbox{}\fi\par}{}{}
\makeatother
% Allow footnotes in longtable head/foot
\IfFileExists{footnotehyper.sty}{\usepackage{footnotehyper}}{\usepackage{footnote}}
\makesavenoteenv{longtable}
\setlength{\emergencystretch}{3em} % prevent overfull lines
\providecommand{\tightlist}{%
  \setlength{\itemsep}{0pt}\setlength{\parskip}{0pt}}
\setcounter{secnumdepth}{-\maxdimen} % remove section numbering
\ifLuaTeX
  \usepackage{selnolig}  % disable illegal ligatures
\fi

\title{Predicción de Rendimiento Estudiantil}
\author{}
\date{}

\begin{document}
\maketitle

\hypertarget{santiago-martuxednez-beltruxe1n}{%
\section{Santiago Martínez
Beltrán}\label{santiago-martuxednez-beltruxe1n}}

\begin{quote}
\emph{Facultad de Ciencias Naturales e Ingeniería}

Universidad Jorge Tadeo Lozano

Bogotá, Colombia

\href{mailto:santiago.martinezb@utadeo.edu.co}{\nolinkurl{santiago.martinezb@utadeo.edu.co}}
\end{quote}

\hypertarget{sergio-daniel-aza-ocampo}{%
\section{\texorpdfstring{\hfill\break
Sergio Daniel Aza
Ocampo}{ Sergio Daniel Aza Ocampo}}\label{sergio-daniel-aza-ocampo}}

\begin{quote}
\emph{Facultad de Ciencias Naturales e Ingeniería}

Universidad Jorge Tadeo Lozano

Bogotá, Colombia

\href{mailto:Sergiod.azaocampo@utadeo.edu.co}{\nolinkurl{Sergiod.azaocampo@utadeo.edu.co}}
\end{quote}

\hypertarget{curso-inteligencia-artificial}{%
\section{Curso: Inteligencia
Artificial}\label{curso-inteligencia-artificial}}

\hypertarget{de-octubre-de-2025}{%
\section{5 de octubre de 2025}\label{de-octubre-de-2025}}

\hypertarget{juliuxe1n-santiago-hernuxe1ndez-gonzuxe1lez}{%
\section{\texorpdfstring{\hfill\break
Julián Santiago Hernández
González}{ Julián Santiago Hernández González}}\label{juliuxe1n-santiago-hernuxe1ndez-gonzuxe1lez}}

\begin{quote}
\emph{Facultad de Ciencias Naturales e Ingeniería}

Universidad Jorge Tadeo Lozano

Bogotá, Colombia

\href{mailto:Julians.hernandezg@uatdeo.edu.co}{\nolinkurl{Julians.hernandezg@uatdeo.edu.co}}
\end{quote}

\hypertarget{section}{%
\section{}\label{section}}

\begin{quote}
\emph{\textbf{Resumen}}

Este proyecto propone el desarrollo de un modelo de aprendizaje
automático capaz de predecir el rendimiento académico de los estudiantes
universitarios a partir de variables socioeducativas y de
comportamiento. El objetivo es identificar factores que influyen en el
desempeño y anticipar posibles riesgos de bajo rendimiento, con el fin
de apoyar estrategias de acompañamiento académico en la Universidad
Jorge Tadeo Lozano.\\
Se utilizará el dataset \emph{Student Performance} del repositorio UCI
Machine Learning {[}1{]}, el cual contiene información de 649
estudiantes de educación media en Portugal.\\
El enfoque de IA corresponde a una tarea de clasificación con modelos
como Árboles de Decisión y Regresión Logística {[}7{]}{[}8{]}{[}9{]}.\\
Se espera lograr una precisión superior al 80\% y comprobar que
variables como las horas de estudio y las ausencias tienen alta
correlación con el rendimiento final {[}4{]}{[}6{]}.

\emph{\textbf{Problema Local y Motivación}}

En el contexto universitario bogotano, el rendimiento académico de los
estudiantes es un factor crítico que influye en la deserción y en el
éxito profesional. En la Universidad Jorge Tadeo Lozano, al igual que en
muchas instituciones, los docentes suelen carecer de herramientas
predictivas que les permitan identificar tempranamente a estudiantes en
riesgo de bajo desempeño.\\
El proyecto busca aplicar inteligencia artificial para apoyar la gestión
académica mediante la predicción del rendimiento a partir de información
disponible, como hábitos de estudio, notas parciales y asistencia
{[}2{]}{[}3{]}{[}5{]}.\\
Esto permitiría orientar intervenciones personalizadas y promover la
permanencia estudiantil, impactando positivamente la calidad educativa y
la eficiencia institucional {[}4{]}.

\emph{\textbf{Dataset}}

Se empleará el \emph{Student Performance Dataset} del repositorio
\textbf{UCI Machine Learning} {[}1{]}. Este conjunto contiene 649
registros de estudiantes y 33 variables, incluyendo factores
demográficos, sociales y académicos.\\
Las variables más relevantes para el modelo son: studytime, failures,
absences, G1, G2 y G3 (notas).\\
El dataset está disponible bajo licencia pública para investigación
educativa y es representativo por su diversidad de características, lo
cual lo hace adecuado para modelar la predicción del rendimiento
académico {[}1{]}{[}4{]}.\\
Los datos se encuentran en formato CSV, lo que facilita su manipulación
y procesamiento mediante bibliotecas de Python como \emph{pandas} y
\emph{scikit-learn} {[}10{]}

\emph{\textbf{Tarea de IA y Algoritmo(s)}}

La tarea corresponde a un problema de \textbf{clasificación supervisada}
en datos tabulares {[}5{]}{[}6{]}.\\
El objetivo del modelo es predecir si un estudiante tendrá un
rendimiento alto o bajo según sus características previas.\\
Se emplearán los algoritmos \textbf{Árbol de Decisión} y
\textbf{Regresión Logística}, por su facilidad de interpretación y
eficiencia en datasets pequeños {[}7{]}{[}8{]}{[}9{]}.\\
El Árbol de Decisión permitirá visualizar los factores determinantes,
mientras que la Regresión Logística ofrecerá una referencia estadística
como línea base {[}7{]}{[}8{]}.

\emph{\textbf{Metodología y Evaluación}}

El proceso inicia con la limpieza del dataset y la codificación de
variables categóricas mediante \emph{Label Encoding} o \emph{One-Hot
Encoding} {[}9{]}.\\
Posteriormente, se dividirán los datos en conjuntos de entrenamiento
(80\%) y prueba (20\%) {[}5{]}.\\
Se evaluará el modelo utilizando métricas como \emph{Accuracy},
\emph{Precision}, \emph{Recall} y \emph{F1-Score}, además de una matriz
de confusión para interpretar los errores {[}9{]}{[}10{]}.\\
Como línea base, se compararán los resultados con un clasificador
aleatorio, verificando si la combinación de árboles de decisión y
regresión logística mejora el rendimiento de la predicción
{[}4{]}{[}6{]}.

\emph{\textbf{Resultados esperados, ética y cronograma}}

Se espera que el modelo prediga con alta exactitud el rendimiento final
y determine qué factores influyen más en el desempeño académico
{[}4{]}{[}6{]}.\\
Los resultados permitirán crear estrategias preventivas en la
universidad, contribuyendo a mejorar la retención y la toma de
decisiones institucionales {[}2{]}{[}3{]}.\\
Desde el punto de vista ético, se garantizará la anonimización de los
datos y el uso exclusivo con fines académicos, evitando sesgos por
género o condición socioeconómica {[}3{]}{[}8{]}.\\
\textbf{Cronograma propuesto:}~
\end{quote}

\begin{itemize}
\item
  \textbf{Semana 1:} Revisión del problema, recolección del dataset y
  exploración inicial de variables.~
\end{itemize}

\begin{itemize}
\item
  \textbf{Semana 2:} Limpieza y preprocesamiento de los datos
  (codificación, normalización y selección de variables).~
\end{itemize}

\begin{itemize}
\item
  \textbf{Semana 3:} Entrenamiento de los modelos (Árbol de Decisión y
  Regresión Logística) y ajuste de parámetros.~
\end{itemize}

\begin{itemize}
\item
  \textbf{Semana 4:} Evaluación del desempeño de los modelos, análisis
  de métricas y visualización de resultados.~
\end{itemize}

\begin{itemize}
\item
  \textbf{Semana 5:} Redacción final del informe, revisión ética,
  corrección de estilo y entrega del paper.~
\end{itemize}

\begin{quote}
\emph{\textbf{Roles de Equipo}}
\end{quote}

\begin{longtable}[]{@{}
  >{\raggedright\arraybackslash}p{(\columnwidth - 4\tabcolsep) * \real{0.3431}}
  >{\raggedright\arraybackslash}p{(\columnwidth - 4\tabcolsep) * \real{0.2878}}
  >{\raggedright\arraybackslash}p{(\columnwidth - 4\tabcolsep) * \real{0.3691}}@{}}
\toprule
\begin{minipage}[b]{\linewidth}\raggedright
\textbf{Rol}
\end{minipage} & \begin{minipage}[b]{\linewidth}\raggedright
\textbf{Integrante}
\end{minipage} & \begin{minipage}[b]{\linewidth}\raggedright
\textbf{Responsabilidad}
\end{minipage} \\
\midrule
\endhead
\textbf{Análisis de datos y preprocesamiento} & Persona 1 & Obtención,
limpieza y análisis del Dataset \\
\textbf{Modelado y métricas} & Persona 2 & Entrenamiento y evaluación de
modelos \\
\textbf{Redacción y ética} & Persona 3 & Elaboración del informe,
revisión ética y cronograma \\
\bottomrule
\end{longtable}

\begin{quote}
\emph{\textbf{Enlace Repositorio de Github:}}

\url{https://github.com/Santiageoff/ProyectoIA.git}
\end{quote}

\textbf{\textsc{Referencias}}

\begin{enumerate}
\def\labelenumi{\arabic{enumi}.}
\item
  P. Cortez and A. M. G. Silva, ``Using data mining to predict secondary
  school student performance,'' University of Minho, Portugal, UCI
  Machine Learning Repository, 2008. {[}Online{]}. Available:
  \url{https://archive.ics.uci.edu/ml/datasets/Student+Performance.}
\item
  C. Romero and S. Ventura, ``Educational data mining and learning
  analytics: An updated survey,'' Wiley Interdisciplinary Reviews: Data
  Mining and Knowledge Discovery, vol. 10, no. 3, e1355, 2020, doi:
  10.1002/widm.1355.
\item
  V. Kumar and A. Chadha, ``An empirical study of the applications of
  data mining techniques in higher education,'' International Journal of
  Advanced Computer Science and Applications (IJACSA), vol. 3, no. 2,
  pp. 80--84, 2012, doi: 10.14569/IJACSA.2012.030214.
\item
  A. N. M. Shahiri, W. Husain, and N. A. Rashid, ``A review on
  predicting student performance using data mining techniques,''
  Procedia Computer Science, vol. 72, pp. 414--422, 2015, doi:
  10.1016/j.procs.2015.12.111.
\item
  S. B. Kotsiantis, C. J. Pierrakeas, and P. E. Pintelas, ``Predicting
  students' performance in distance learning using machine learning
  techniques,'' Applied Artificial Intelligence, vol. 18, no. 5, pp.
  411--426, 2004, doi: 10.1080/08839510490442058.
\item
  S. Huang, N. Fang, and N. Chen, ``Predicting student academic
  performance with educational data mining,'' Frontiers in Education
  Technology, vol. 1, no. 1, pp. 1--8, 2018, doi: 10.22158/fet.v1n1p1.
\item
  M. Kumar and A. Singh, ``Predicting student performance using decision
  tree algorithms,'' International Journal of Computer Applications,
  vol. 175, no. 5, pp. 22--26, 2017, doi: 10.5120/ijca2017914605.
\item
  S. Raschka and V. Mirjalili, *Python Machine Learning: Machine
  Learning and Deep Learning with Python, scikit-learn, and TensorFlow
  2*, 4th ed. Birmingham, UK: Packt Publishing, 2022. {[}Online{]}.
  Available:
  \url{https://www.packtpub.com/product/python-machine-learning-fourth-edition/9781801819312}.
\item
  A. Géron, *Hands-On Machine Learning with Scikit-Learn, Keras and
  TensorFlow*, 3rd ed. Sebastopol, CA: O'Reilly Media, 2022.
  {[}Online{]}. Available:
  \url{https://www.oreilly.com/library/view/hands-on-machine-learning/9781098125967/}.
\item
  F. Pedregosa, G. Varoquaux, A. Gramfort, V. Michel, B. Thirion, O.
  Grisel, M. Blondel, P. Prettenhofer, R. Weiss, V. Dubourg, J.
  Vanderplas, A. Passos, D. Cournapeau, M. Brucher, M. Perrot, and E.
  Duchesnay, ``Scikit-learn: Machine learning in Python,'' *Journal of
  Machine Learning Research*, vol. 12, pp. 2825--2830, 2011.
  {[}Online{]}. Available:
  \url{https://jmlr.org/papers/v12/pedregosa11a.html}.
\end{enumerate}

\end{document}
